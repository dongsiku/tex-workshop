\documentclass[11pt,a4paper]{jsarticle}
%
\usepackage{amsmath,amssymb}
\usepackage{bm}
\usepackage[dvipdfmx]{graphicx}
\usepackage{ascmac}
\usepackage{titlesec}
\usepackage{otf}
\usepackage{amsthm}
\usepackage{physics}
\usepackage{booktabs}
%
\setlength{\textwidth}{\fullwidth}
\setlength{\textheight}{39\baselineskip}
\addtolength{\textheight}{\topskip}
%\setlength{\voffset}{-0.1in}
\setlength{\topmargin}{0pt}
\setlength{\headheight}{0pt}
\setlength{\headsep}{0pt}

% \renewcommand{\thesection}{演習問題\arabic{section}}
% \titleformat*{\section}{\normalsize\bfseries}
% \titleformat*{\subsection}{\normalsize\bfseries}

\newtheorem{problem}{演習問題}
\newtheorem{answer}{解答}
\renewcommand{\theanswer}{}
\newcommand{\N}{\mathcal{N}}
%
%\newcommand{\divergence}{\mathrm{div}\,}  %ダイバージェンス
%\newcommand{\grad}{\mathrm{grad}\,}  %グラディエント
%\newcommand{\rot}{\mathrm{rot}\,}  %ローテーション
\newcommand{\Zahlen}{\mathbb{Z}}
\newcommand{\const}{\mathrm{const}}
\newcommand{\laplace}{\mathcal{L}}

\begin{document}

\begin{center}
  {\Large\bfseries \TeX 演習問題} % title
\end{center}
\begin{flushright}
  {\large 43M21224 諏訪 棟植} % 学生番号と氏名と氏名を書き換えてください
\end{flushright}

% 提出するファイル名は,practice_YourName.pdfにしてください.
% 例:practice_SuwaMuneue.pdf
%
% 大文字のデルタや円周率パイなど,不明なギリシャ文字はググってください.
%
% 図は,パワポなどで作成した後,PDFファイルで保存したうえで,
% pdfcropコマンドでトリミングしてください.
% なお,図の問題は,pdfファイルの図を挿入することを目的として出題しているため,
% 図自体は適当でかまいません.

\section{数学}

\subsection{フーリエ級数展開}

% ここから編集してください

% ここまで編集してください

\end{document}
