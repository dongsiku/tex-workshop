\documentclass[11pt,a4paper]{jsarticle}
%
\usepackage{amsmath,amssymb}
\usepackage{bm}
\usepackage[dvipdfmx]{graphicx}
\usepackage{ascmac}
\usepackage{url}
%
\usepackage{titlesec}     % For \titleformat
% \usepackage{enumerate}  % For enumerate
% \usepackage{ulem}       % For underlining
%
\titleformat*{\section}{\large\bfseries}
\titleformat*{\subsection}{\normalsize\bfseries}
\setcounter{section}{-1}
%
\setlength{\textwidth}{\fullwidth}
\setlength{\textheight}{37\baselineskip}
\addtolength{\textheight}{\topskip}
\setlength{\voffset}{-0.5in}
\setlength{\headsep}{0.3in}
%
%\newcommand{\divergence}{\mathrm{div}\,}  %ダイバージェンス
%\newcommand{\grad}{\mathrm{grad}\,}  %グラディエント
%\newcommand{\rot}{\mathrm{rot}\,}  %ローテーション
%
\pagestyle{myheadings}
\markright{知能機械制御学研究室:\TeX 講習会の資料}
\begin{document}
%
%
\section{はじめに}

この文書は,\TeX の文法について,数式を中心にざっくりとまとめたものである.本文書も\TeX で書かれているため,必要に応じて\TeX ファイル``tex-workshop.tex''を参照せよ.なお,詳細は``tex ギリシャ文字''などでググること.

\section{数式の基本}

\subsection{文中の数式}

文中の数式$j=\sqrt{-1}$は,\verb|$j=\sqrt{-1}$|といったように,\verb|$|で囲んで書く.

\subsection{別行立ての数式}

別行立ての数式は,\verb|\begin{align},\end{align}|といったように,``align''で囲んで書く.また,2行以上の数式の場合は,\verb|&|を用いることで,文字を揃えられる.なお,数式中の空白文字は無視されるため,必要な時は\verb|~|や\verb|\quad|などを用いる.
\begin{align}
  x^3+y^3             & =(x+y)\left(x^2-xy+y^2\right) \label{factorization-1} \\
  x^3-y^3             & =(x-y)\left(x^2+xy+y^2\right) \label{factorization-2} \\
  \frac{x^3-y^3}{x-y} & =x^2+xy+y^2,~(x\neq y) \label{factorization-3}
\end{align}

\subsection{式番号}

ここで,\eqref{factorization-1}などと数式を指定する場合は,式に\verb|\label{factorization-1}|などのラベル付けたうえで,\verb|\eqref{factorization-1}|のように``eqref''を使う.数式の番号が一部のみ必要なときは,\eqref{factorization-4}のように\verb|\nonumber|を用いる.
\begin{align}\label{factorization-4}
  4x^2+8x+4 & =4\left(x^2+2x+1\right)\nonumber \\
            & =4(x+1)^2
\end{align}
まったく必要ないときは,\verb|\begin{align*},\end{align*}|で囲う.
\begin{align*}
  x^3-1=(x-1)\left(x^2+x+1\right)
\end{align*}

\subsection{その他}

これ以外にも,
\begin{equation*}
  \delta(x) =
  \begin{cases}
    1, & x \geq 0 \\
    0, & x < 0
  \end{cases}
\end{equation*}
のように場合分けの数式なども記述可能である.詳細は,``amsmathの数式環境まとめ''(\url{https://qiita.com/t_kemmochi/items/a4c390b4967b13f3afb7})などを参照せよ.

\section{下付き文字と上付き文字}

下付き文字$z_1<1$は,\verb|$z_1<1$|のように\verb|_|を,上付き文字は$j^2=-1$のように\verb|^|を用いる.なお,下付き文字や上付き文字が2文字以上$a_{10},b^{20},c_{10}^{20}$のときは\verb|$a_{10},b^{20},c_{10}^{20}$|のように\verb|{ }|を用いる.

\section{和・積分(シグマとインテグラル)}

和(シグマ)は\verb|\sum|(\verb|\Sigma|でないことに注意せよ),積分(インテグラル)は\verb|\int|を用いる.下限と上限は,下付き文字と上付き文字のように\verb|_{}^{}|を用いる.
\begin{align*}
  \sum_{k=1}^{n}a_k & =a_1+a_2+a_3+\cdots+a_n \\
  \int_0^1 x dx     & = \frac{1}{2}
\end{align*}

\section{分数}

分数は基本的に\verb|\frac|を用いる.ただし,テキストスタイルとディスプレイスタイルは,それぞれ\verb|\tfrec,\dfrac|をもちいる.\verb|\frec{K}{Ts+1},\tfrec{K}{Ts+1},\dfrac{K}{Ts+1}|について,文中の数式ではそれぞれ$\frac{K}{Ts+1},\tfrac{K}{Ts+1},\dfrac{K}{Ts+1}$と,別行立ての数式では
\begin{align*}
  G(s) & =\frac{K}{Ts+1},  \\
  G(s) & =\tfrac{K}{Ts+1}, \\
  G(s) & =\dfrac{K}{Ts+1}
\end{align*}
と表記される.

\end{document}
